\documentclass[11pt]{article}

\usepackage[utf8]{inputenc}
\usepackage[T1]{fontenc}
\usepackage{listings}
\usepackage{geometry}
\usepackage{hyperref}
\usepackage{xcolor}
\setlength{\parindent}{0pt}
\geometry{top = 0.5in, bottom = 0.5 in, left = 0.5 in, right = 0.5 in}
\tolerance=1000
\author{Lisa Breckels and Laurent Gatto}
\date{\today}

\title{A Bioconductor workflow for processing and analysing spatial proteomics data \\ Reply to reviewers}

\begin{document}

\maketitle

\section*{Reviewer 1 - Daniel J. Stekhoven}

We thank reviewer 1 for their comments. Please find our responses to
these inset below.

\begin{quote} \textcolor{gray}{ Next to reducing the dimensions of
data for visualisation, PCA also offers a way to understand how the
variability is distributed across the multidimensional data by
providing linear combinations of the variables which then constitute
the actual PCs. On that note it would be nice to mention this in
Visualising markers section on page 16, where PC7 explains not much
variability but due to the correct weighing of the variables we do
get a separation between mitochondrial and peroxisome. This then can
be further motivated with Figure 9 - where we probably can see that
the weights for the fractions where the two localisations differ are
larger than otherwise.} \end{quote}

We have added a paragraph reminding users what PCA does in the
Visualising Markers section to motivate the choice of looking at
PC's 1 and 7. Figure 9 (now Figure 8), the corresponding code and an
explanation of the plotDist function has been moved to this section
to lead on from the reference to marker profiles and separation.  

\begin{quote}  \textcolor{gray}{ I was unable to reproduce Figure 13
comparing the two MSnSets. While I was able to look at each set
separately using pRolocVis(hllst@x[[I]]), where i is 1 or 2, I only
got an error using the code from the manuscript. When using
?remap=FALSE? it actually works, but since this makes barely sense
it is of no use - but just as a hint at debugging it.  } \end{quote}

This should be fixed in the latest version of \texttt{pRolocGUI}.


\begin{quote}  \textcolor{gray}{ You really need to make the results
from the phenoDisco classification available too. It is super
disappointing that one cannot continue reproducing the code from
page 23 on, because it takes 24 hours to compute it using 40 cores?
} \end{quote}

The results are available and stored in `pRolocdata` for users. This
is what is called in the manuscript under the hood:

\begin{lstlisting}[language=R] extdatadir <- system.file("extdata",
package = "pRolocdata") csvfile <- dir(extdatadir, full.names =
TRUE, pattern = "hyperLOPIT-SIData-ms3-rep12-intersect.csv") f0 <-
dir(extdatadir, full.names = TRUE, pattern = "bpw-pdres.rds") pdres
<- readRDS(f0) hl <- addMarkers(hl, pdres, mcol = "pd", verbose =
FALSE) \end{lstlisting}

Please note that we already say in the workflow: "Note: We do not
evaluate this code chunk in this document as the algorithm is
computationally intensive and best parallelised over multiple
workers. This phenoDisco analysis took 24 hours to complete when
parallelised over 40 workers." \newline \\ However, we have moved
this statement to the end of the section for clarity. We state the
output of these analyses is readily available for readers and stored
in the \texttt{pd} column of the \texttt{MSnSet} object called
\texttt{hl}. Readers can examine these results and get an idea of
the type of output one can expect from this a \texttt{phenoDisco}
analysis.  \newline \\ Specifically we have added the following to
the end of the section to make this clear.  \newline \\ "Please
note, in this document we can not evaluate the call to
\texttt{phenoDisco} in the code chunk above as the algorithm is
computationally intensive and best parallelised over multiple
workers. This phenoDisco analysis took ~24 hours to complete when
parallelised over 40 workers.  The ouput of running the
\texttt{phenoDisco} algorithm is an \texttt{MSnSet} containing the
new data clusters, appended to the \texttt{featureData} under the
name \texttt{pd}. We have made the results readily available for
users to interrogate and get an idea of the type of output one may
gain from using this function. The results can be displayed by using
the \texttt{getMarkers} function. We see that 5 new phenotype data
clusters were found."


\begin{quote}  \textcolor{gray}{ The above comment is of course also
true for the KNN TL Optimisation on page 33 - this needs to be
downloadable, since not everyone has access to Cambridge?s HPC and
probably even less have 76 hours to spare.} \end{quote}

The same as for the phenoDisco analysis and svm, the TL results are
stored as a RDS in pRolocdata and loaded in the backgroun


We thank the reviewer for their constructive and positive comments.

\section*{Reviewer 2: Leonard J. Foster}

\begin{quote}\textcolor{gray}{ } \end{quote}


\end{document}
